\documentclass[acmsmall,review,anonymous]{acmart}

\def\techreport{1}

\usepackage{amsmath}
\usepackage{amsthm}

\usepackage{url}
\usepackage{listings, lstautogobble}
\usepackage{xspace}

\usepackage{hyperref}
\def\sectionautorefname{Section}
\def\subsectionautorefname{Section}

\if\techreport1
    \newcommand{\bsqon}{\textsc{BsqON}\xspace}
    \newcommand{\srclocation}{\url{https://github.com/BosqueLanguage/BSQON}}
\else
    \newcommand{\bsqon}{\textsc{DataON}\xspace}
    \newcommand{\srclocation}{Omitted For Review}
\fi


\newcommand{\todo}[1]{{\color{red}#1}}

\newcommand{\eg}{\hbox{\emph{e.g.}}\xspace}
\newcommand{\ie}{\hbox{\emph{i.e.}}\xspace}
\newcommand{\etc}{\hbox{\emph{etc.}}\xspace}
\newcommand{\vs}{\hbox{\emph{vs.}}\xspace}
\newcommand{\st}{\hbox{\emph{s.t.}}\xspace}
\newcommand{\wrt}{\hbox{\emph{w.r.t.}}\xspace}

\newcommand{\cf}[1]{\texttt{#1}}

\newcommand\bnfalt{\;\;|\;\;}
\newcommand\bnfas{\;\;:=\;\;}

\newtheorem{theorem}{Theorem}

\usepackage{color}
\definecolor{purple}{RGB}{75, 0, 255}
\definecolor{cgreen}{rgb}{0.25,0.5,0.35} % comments

%define bosque language
\lstdefinelanguage{bosque}{
keywords={concept, entity, datatype, typedecl, provides, field, switch, match, abstract, method, if, then, elif, else, function, return, true, false, none, let, var, in, requires, ensures, invariant, validate, recursive, using, of, this, $, pred, fn, ref, examples, for, defer, test, const, override, $return},
keywordstyle=\color{blue}\bfseries,
identifierstyle=\color{black},
alsoother={@},
sensitive=true,
comment=[l]{//},
morecomment=[s]{/*}{*/},
commentstyle=\color{cgreen}\bfseries,
stringstyle=\color{red}\ttfamily
}
 
\lstset{
language=bosque,
extendedchars=true,
basicstyle=\scriptsize\ttfamily,
showstringspaces=false,
showspaces=false,
numbers=none,
numberstyle=\footnotesize,
numbersep=9pt,
tabsize=2,
breaklines=true,
showtabs=false,
captionpos=b,
autogobble=true
}

\markright{XXXX}

%% Rights management information.  This information is sent to you
%% when you complete the rights form.  These commands have SAMPLE
%% values in them; it is your responsibility as an author to replace
%% the commands and values with those provided to you when you
%% complete the rights form.
\setcopyright{acmlicensed}
\copyrightyear{2018}
\acmYear{2018}
\acmDOI{XXXXXXX.XXXXXXX}


%%
%% These commands are for a JOURNAL article.
\acmJournal{JACM}
\acmVolume{37}
\acmNumber{4}
\acmArticle{111}
\acmMonth{8}

%%
%% Submission ID.
%% Use this when submitting an article to a sponsored event. You'll
%% receive a unique submission ID from the organizers
%% of the event, and this ID should be used as the parameter to this command.
%%\acmSubmissionID{123-A56-BU3}

%%
%% For managing citations, it is recommended to use bibliography
%% files in BibTeX format.
%%
%% You can then either use BibTeX with the ACM-Reference-Format style,
%% or BibLaTeX with the acmnumeric or acmauthoryear sytles, that include
%% support for advanced citation of software artefact from the
%% biblatex-software package, also separately available on CTAN.
%%
%% Look at the sample-*-biblatex.tex files for templates showcasing
%% the biblatex styles.
%%

%%
%% The majority of ACM publications use numbered citations and
%% references.  The command \citestyle{authoryear} switches to the
%% "author year" style.
%%
%% If you are preparing content for an event
%% sponsored by ACM SIGGRAPH, you must use the "author year" style of
%% citations and references.
%% Uncommenting
%% the next command will enable that style.
%%\citestyle{acmauthoryear}

\begin{document}

%%
%% The "title" command has an optional parameter,
%% allowing the author to define a "short title" to be used in page headers.
\title{Better Regexing with BREX -- A Modern Regex DSL}

\author{Mark Marron}
\email{marron@cs.uky.edu}
%\orcid{1234-5678-9012}
\affiliation{%
  \institution{University of Kentucky}
  \city{Lexington}
  \state{Kentucky}
  \country{USA}
}

%%
%% The abstract is a short summary of the work to be presented in the
%% article.
\begin{abstract}
Regular expressions (regexs) are a powerful and foundational programming features for string processing. Every mainstream language provides 
builtin, and often specialized support for them. Emperical evidence suggests that regexs are widely used, appearing at least once in 
one third of Python and JavaScript projects.  and are a critical tool for many

Despite this importance, regexs are often difficult to write, read, and maintain. Developers often refer to them as ''read-only'' and 
experience shows they are frequently the source of buggy application behavior. 
Apochrphally -- you may think to yourself I have a problem, I'll use regexs, now I have two problems. 

Despite these issues the DSL and core semantics of regexs in programming languages have remained largely unchanged since the 
introduction of the xxx in the kkk and minorly extended into yyyy (PCRE) in the jjjs.
\end{abstract}

%%
%% The code below is generated by the tool at http://dl.acm.org/ccs.cfm.
%% Please copy and paste the code instead of the example below.
%%
\begin{CCSXML}
<ccs2012>
 <concept>
  <concept_id>00000000.0000000.0000000</concept_id>
  <concept_desc>Do Not Use This Code, Generate the Correct Terms for Your Paper</concept_desc>
  <concept_significance>500</concept_significance>
 </concept>
</ccs2012>
\end{CCSXML}

\ccsdesc[500]{Do Not Use This Code~Generate the Correct Terms for Your Paper}

\keywords{Keywords!!!}

%\received{20 February 2007}
%\received[revised]{12 March 2009}
%\received[accepted]{5 June 2009}

%%
%% This command processes the author and affiliation and title
%% information and builds the first part of the formatted document.
\maketitle

\cite{bosque}

\section{Introduction}
The usual

\section{PCRE Review}

\section{Better Regexing}

\subsection{Regex DSL Syntax}

\subsection{Regex DSL Semantics}

\subsection{Regex Engine}

\section{Better Globbing}

\subsection{Glob DSL Syntax}

\subsection{Glob DSL Semantics}

\subsection{Glob Engine}

\section{Case Studies}

\subsection{Validation}

\subsection{Parsing}

\subsection{Globbing}

\section{Related Work}

\section{Onward!}

\section*{Data-Availability Statement}
This....


%\begin{acks}
%To Robert, for the bagels and explaining CMYK and color spaces.
%\end{acks}

\bibliographystyle{ACM-Reference-Format}
\bibliography{bibfile}


\end{document}
\endinput