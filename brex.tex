\documentclass[sigplan,10pt,review]{acmart}
\settopmatter{printfolios=true,printccs=false,printacmref=false}

%% For final camera-ready submission, w/ required CCS and ACM Reference
%\documentclass[sigplan]{acmart}\settopmatter{}

%% Conference information
%% Supplied to authors by publisher for camera-ready submission;
%% use defaults for review submission.
\acmConference[PL'18]{ACM SIGPLAN Conference on Programming Languages}{January 01--03, 2018}{New York, NY, USA}
\acmYear{2018}
\acmISBN{} % \acmISBN{978-x-xxxx-xxxx-x/YY/MM}
\acmDOI{} % \acmDOI{10.1145/nnnnnnn.nnnnnnn}
\startPage{1}

\def\techreport{1}

\usepackage{amsmath}
\usepackage{amsthm}

\usepackage{url}
\usepackage{listings, lstautogobble}
\usepackage{xspace}

\usepackage{hyperref}
\def\sectionautorefname{Section}
\def\subsectionautorefname{Section}

\if\techreport1
    \newcommand{\bsqon}{\textsc{BsqON}\xspace}
    \newcommand{\srclocation}{\url{https://github.com/BosqueLanguage/BSQON}}
\else
    \newcommand{\bsqon}{\textsc{DataON}\xspace}
    \newcommand{\srclocation}{Omitted For Review}
\fi


\newcommand{\todo}[1]{{\color{red}#1}}

\newcommand{\eg}{\hbox{\emph{e.g.}}\xspace}
\newcommand{\ie}{\hbox{\emph{i.e.}}\xspace}
\newcommand{\etc}{\hbox{\emph{etc.}}\xspace}
\newcommand{\vs}{\hbox{\emph{vs.}}\xspace}
\newcommand{\st}{\hbox{\emph{s.t.}}\xspace}
\newcommand{\wrt}{\hbox{\emph{w.r.t.}}\xspace}

\newcommand{\cf}[1]{\texttt{#1}}

\newcommand\bnfalt{\;\;|\;\;}
\newcommand\bnfas{\;\;:=\;\;}

\newtheorem{theorem}{Theorem}

\usepackage{color}
\definecolor{purple}{RGB}{75, 0, 255}
\definecolor{cgreen}{rgb}{0.25,0.5,0.35} % comments

%define bosque language
\lstdefinelanguage{bosque}{
keywords={concept, entity, datatype, typedecl, provides, field, switch, match, abstract, method, if, then, elif, else, function, return, true, false, none, let, var, in, requires, ensures, invariant, validate, recursive, using, of, this, $, pred, fn, ref, examples, for, defer, test, const, override, $return},
keywordstyle=\color{blue}\bfseries,
identifierstyle=\color{black},
alsoother={@},
sensitive=true,
comment=[l]{//},
morecomment=[s]{/*}{*/},
commentstyle=\color{cgreen}\bfseries,
stringstyle=\color{red}\ttfamily
}
 
\lstset{
language=bosque,
extendedchars=true,
basicstyle=\scriptsize\ttfamily,
showstringspaces=false,
showspaces=false,
numbers=none,
numberstyle=\footnotesize,
numbersep=9pt,
tabsize=2,
breaklines=true,
showtabs=false,
captionpos=b,
autogobble=true
}

\markright{XXXX}

%% Rights management information.  This information is sent to you
%% when you complete the rights form.  These commands have SAMPLE
%% values in them; it is your responsibility as an author to replace
%% the commands and values with those provided to you when you
%% complete the rights form.
\setcopyright{none}

%%
%% Submission ID.
%% Use this when submitting an article to a sponsored event. You'll
%% receive a unique submission ID from the organizers
%% of the event, and this ID should be used as the parameter to this command.
%%\acmSubmissionID{123-A56-BU3}

\begin{document}

%% The "title" command has an optional parameter,
%% allowing the author to define a "short title" to be used in page headers.
\title{Better Regexing with BREX -- A Modern Regex DSL}

\author{Mark Marron}
\email{marron@cs.uky.edu}
%\orcid{1234-5678-9012}
\affiliation{%
  \institution{University of Kentucky}
  \city{Lexington}
  \state{Kentucky}
  \country{USA}
}

%%
%% The abstract is a short summary of the work to be presented in the
%% article.
\begin{abstract}
Regular expressions (regexps) are a powerful and foundational tool for string progressing in programming languages. Almost without 
exception, every modern programming language provides regex support as a primitive builtin. Empirical evidence indicates that these 
features are widely and frequently used, \ie at least once in one third of Python or JavaScript projects. Despite this importance, 
regexps are often difficult to write, read, and maintain. Developers often refer to them as ''read-only'' and experience shows they 
are frequently the source of buggy application behavior. 

In contrast to other language features, which have responded to high-value \& universally acknowleged difficulties with use by experimenting 
with altering syntax and semantics for a feature, the sub-language for regexps is (almost) universal across programming languages. Languages 
as diverse as Java, C++, Python, JavaScript, Haskell, all provide an almost identical (POSIX/PCRE) regex sub-language first introduced in 
the 1980's (POSIX) and 1990's (PCRE)! This paper presents a new regex sub-language, BREX, which (1) addresses many of the issues identified 
in existing regex languages/implementations and (2) introduce features that support the demands on structured text processing in modern 
applications and programming languages.
\end{abstract}

%%
%% The code below is generated by the tool at http://dl.acm.org/ccs.cfm.
%% Please copy and paste the code instead of the example below.
%%
\begin{CCSXML}
<ccs2012>
 <concept>
  <concept_id>00000000.0000000.0000000</concept_id>
  <concept_desc>Do Not Use This Code, Generate the Correct Terms for Your Paper</concept_desc>
  <concept_significance>500</concept_significance>
 </concept>
</ccs2012>
\end{CCSXML}

\ccsdesc[500]{Do Not Use This Code~Generate the Correct Terms for Your Paper}

\keywords{Keywords!!!}

%\received{20 February 2007}
%\received[revised]{12 March 2009}
%\received[accepted]{5 June 2009}

%%
%% This command processes the author and affiliation and title
%% information and builds the first part of the formatted document.
\maketitle

\cite{bosque}

\section{Introduction}
The usual

Despite this importance, and the variety of design choices in other aspects of programming languages, the regex 
subset  


  Regular expressions (regexs) are a powerful and foundational programming features for string processing. Every mainstream language provides 
builtin, and often specialized support for them. Emperical evidence suggests that regexs are widely used, appearing at least once in 
one third of Python and JavaScript projects.  and are a critical tool for many

Despite this importance, regexs are often difficult to write, read, and maintain. Developers often refer to them as ''read-only'' and 
experience shows they are frequently the source of buggy application behavior. 
Apochrphally -- you may think to yourself I have a problem, I'll use regexs, now I have two problems. 

Despite these issues the DSL and core semantics of regexs in programming languages have remained largely unchanged since the 
introduction of the xxx in the kkk and minorly extended into yyyy (PCRE) in the jjjs.

\section{Better Regexing}

\subsection{Regex Syntax}

\subsection{Regex Semantics}

\subsection{Regex Engine}

\section{Paths and Globs}

\subsection{Path \& Glob Syntax}

\subsection{Glob Semantics}

\subsection{Glob Engine}

\section{Case Studies}

\subsection{Validation}

\subsection{Parsing}

\subsection{Globbing}

\section{Related Work}

\section{Onward!}

%\begin{acks}
%To Robert, for the bagels and explaining CMYK and color spaces.
%\end{acks}

\bibliographystyle{ACM-Reference-Format}
\bibliography{bibfile}


\end{document}
