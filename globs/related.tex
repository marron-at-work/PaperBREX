%!TEX root = globs.tex

Studies such as~\cite{regexes-are-hard,type-system-regex,regex-usage-and-context,regex-comprehension,regex-test-in-wild,regexBugs,regex-evolution} explore the challenges developers 
face with pattern-matching tools, particularly regular expressions (regex). These studies highlight common pitfalls, such as incorrect pattern matching, performance issues, and 
security vulnerabilities like ReDoS (Regular Expression Denial of Service)~\cite{impact-of-ReDoS, rethink-regex}. While regex has been extensively studied, globbing has received 
less attention despite its widespread use in shell, scripting, build tools, and configuration management.

Our research extends these findings by focusing specifically on globbing's usability, portability, and shortcomings across programming environments. By analyzing real-world usage 
patterns and developer discussions, we identify gaps in globbing implementations and propose solutions to improve cross-platform compatibility and developer productivity. The 
semantics of pattern-matching tools vary significantly across programming languages and platforms. For example~\cite{disambiguation-of-regex} proposed an analysis for checking if 
a regex was sensitive to greedy/lazy match semantics. Globbing, as a specialized form of pattern matching, exhibits similar variability. These inconsistencies pose challenges 
for developers writing scripts. Our work builds on these studies by documenting the differences in globbing and proposing a standardized approach to improve consistency.

File system operations, including path manipulation and pattern matching, are fundamental to software development. Research such as~\cite{file-system-evolution} discusses the 
evolution of file systems and the semantics bugs across file systems. \cite{ACID} explored extending ACID semantics to file systems, emphasizing the importance of consistency 
and reliability in file handling. Globbing plays a critical role in file system operations, enabling developers to match and manipulate file paths. However, the lack of standardization 
in globbing implementations leads to portability issues, particularly when scripts are moved between Unix-like systems and Windows. Our research addresses these challenges by 
analyzing globbing behavior across platforms and proposing enhancements to improve reliability and portability. 

Security vulnerabilities in pattern-matching tools, such as path traversal attacks and ReDoS~\cite{ReDoScase} in regular expressions, have been widely documented. Globbing is not 
immune to these issues. Attacks in the wild that use improper validation of glob patterns, can lead to path traversal attacks, or globbing introduced performance 
attacks~\cite{insecureparsing, heapoutofbound,stackexhaustion}. Our research examines these security implications and proposes best practices for secure glob patterns. 

Cross-platform compatibility is a recurring challenge in software development. These challenges are exacerbated by inconsistent behavior of globbing, such as differences in 
case sensitivity and path separator handling, \eg~\cite{caseinsensitive} and \cite{pathescape}. Our study builds on this work by empirically analyzing globbing behavior across 
multiple platforms and programming languages, identifying common sources of incompatibility, and proposing solutions to improve portability.
